% !TEX TS-program = xelatex
% !TEX encoding = UTF-8 Unicode
% !Mode:: "TeX:UTF-8"

\documentclass{REsume}

\begin{document}

\name{孔夫子}

\contactInfo{(+86) 114 5141 9198}{potune2@une.gov}{地球联合国前总统}{地球联合国首席科学家}

\section{个人总结}

本人毕业于中国人民大学鲁克思主义学院,拥有鲁克思主义原理本科和硕士学位;
在校学习成绩优异,担任多个学生组织领导职务,在鲁克思主义物种观课题组中,取得鲁克思主义创新大赛特等奖。
毕业后,当选地球联合国第二任总统,卸任后负责 USS 旅行者科研船,后担任地球联合国首席科学家。
本人在职期间,先后获得诺贝尔和平奖、鲁克思主义终身贡献表彰大会特等奖等奖项,在社会学研究领域取得多个重大突破。

\section{教育背景}

\datedsubsection{\textbf{中国人民大学},鲁克思主义原理,\textit{本科}}{2198.9.1 - 2202.6.30}

在校期间担任鲁克思主义学院学生会主席,校鲁克思研究社评论员,校社会学部学工学生负责人,获得校优秀学生干部称号,获得校优秀学生奖学金。

\datedsubsection{\textbf{中国人民大学},鲁克思主义原理,\textit{硕士研究生}}{2202.9.1 - 2205.6.30}

在校期间参加鲁克思主义物种观课题组,研究成果获得鲁克思主义创新大赛特等奖,曾兼任地球联合国外交部特使助理。

\datedsubsection{\textbf{中国人民大学},鲁克思主义原理,\textit{在职博士研究生}}{2205.9.1 - 2210.6.30}

担任地球联合国总统期间,完成在职博士研究生课程课题,并获得在职博士学位和荣誉博士学位。

\section{技术能力}

\begin{itemize}
  \item \textbf{10 级能力:}为一名 10 级领袖,具有更快的研究和调查速度,可调查高级异常现象。
  \item \textbf{鲁克思转世:}命定特质;鲁克思后第一人,在社会学研究领域具有超凡的创造力。
  \item \textbf{科研领头人:}内阁特质;增加帝国研究产出。
\end{itemize}

\section{研究课题}

\datedsubsection{\textbf{鲁克思主义物种观},中国人民大学鲁克思学院}{2202.9.1 - 2205.6.30}

\datedsubsection{\textbf{鲁克思主义星系外交},地球联合国}{2205.10.25 - 2210.10.25}

\datedsubsection{\textbf{鲁克思主义与社会学研究},地球联合国科学部}{2210.11.1 至今}

\section{工作经历}

\datedsubsection{\textbf{地球联合国},总统}{2205 - 2210}

\begin{itemize}
  \item 与首个外星帝国建立外交关系;发起星海共同体倡议。
\end{itemize}

\datedsubsection{\textbf{USS 旅行者},科学家}{2205 - 2210}

\begin{itemize}
  \item 完成多个先驱者异常现象调查;发现赛博勒克斯母星。
\end{itemize}

\datedsubsection{\textbf{地球联合国},首席科学家}{2215 至今}

\begin{itemize}
  \item 完成多项重要科技研究。
\end{itemize}

\section{竞赛获奖}

\datedsubsection{\textbf{鲁克思主义创新大赛},特等奖}{2208.10}

\datedsubsection{\textbf{诺贝尔奖},和平奖}{2208.10}

\datedsubsection{\textbf{鲁克思主义终身贡献表彰大会},特等奖}{2208.10}

\end{document}
